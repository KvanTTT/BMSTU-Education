\usepackage[utf8x]{inputenc}
\usepackage[T2A]{fontenc}
% \usepackage[russian, english]{babel}
\usepackage[english,russian]{babel}

% Опционно, требует  apt-get install scalable-cyrfonts.*
% и удаления одной строчки в cyrtimes.sty
% Сточку не удалять!
% \usepackage{cyrtimes}

% Картнки и tikz
\usepackage{graphicx}

% Некоторая русификация.
\usepackage{misccorr}
\usepackage{amsmath}
\usepackage{indentfirst}
\renewcommand{\labelitemi}{\normalfont\bfseries{--}}
% Увы, костыль
% \addto\captionsrussian{\def\figurename{{\cyr\CYRR\cyri\cyrs\cyru\cyrn\cyro\cyrk}}}
 


% Листинги

\usepackage{listingsutf8} 
\lstset{ 
inputencoding=utf8/koi8-r,
basicstyle=\small,
numbers=left,
numberstyle=\small,
stepnumber=1,
showspaces=false,
showtabs=false,
frame=single,
tabsize=4,
% captionpos=b,
breaklines=true,
% breakatwhitespace=false,
% escapeinside={\%*}{*)}
showstringspaces=false
}
\renewcommand{\lstlistingname}{Листинг}


% Увы, поля придётся уменьшить из-за листингов.
\topmargin -1cm
\oddsidemargin -0.5cm
\evensidemargin -0.5cm
\textwidth 17cm
\textheight 24cm

\sloppy

% Для исходного кода в тексте
\newcommand{\Code}[1]{\texttt{#1}}

