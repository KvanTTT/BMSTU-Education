\documentclass[a4paper,12pt]{article}

\usepackage[utf8x]{inputenc}
\usepackage[T2A]{fontenc}
% \usepackage[russian, english]{babel}
\usepackage[english,russian]{babel}

% Опционно, требует  apt-get install scalable-cyrfonts.*
% и удаления одной строчки в cyrtimes.sty
% Сточку не удалять!
% \usepackage{cyrtimes}

% Картнки и tikz
\usepackage{graphicx}

% Некоторая русификация.
\usepackage{misccorr}
\usepackage{amsmath}
\usepackage{indentfirst}
\renewcommand{\labelitemi}{\normalfont\bfseries{--}}
% Увы, костыль
% \addto\captionsrussian{\def\figurename{{\cyr\CYRR\cyri\cyrs\cyru\cyrn\cyro\cyrk}}}
 


% Листинги

\usepackage{listingsutf8} 
\lstset{ 
inputencoding=utf8/koi8-r,
basicstyle=\small,
numbers=left,
numberstyle=\small,
stepnumber=1,
showspaces=false,
showtabs=false,
frame=single,
tabsize=4,
% captionpos=b,
breaklines=true,
% breakatwhitespace=false,
% escapeinside={\%*}{*)}
showstringspaces=false
}
\renewcommand{\lstlistingname}{Листинг}


% Увы, поля придётся уменьшить из-за листингов.
\topmargin -3cm
\oddsidemargin -0.5cm
\evensidemargin -0.5cm
\textwidth 18cm
\textheight 27cm

\sloppy

% Для исходного кода в тексте
\newcommand{\Code}[1]{\texttt{#1}}


\begin{document}

\begin{titlepage}
\newpage



\begin{center}
\Large Отчет по лабораторным работам 11-12 \\ по дисциплине: <<Методы вычислений>> \\ Вариант 8
\end{center}

\vspace{12em}

\begin{center}
\textsc{\textbf{Минимизация функций двух переменных \\при наличии ограничений.}}
\end{center}

\vspace{27em}

\begin{flushleft}
Студент\hrulefill Кочуркин И.А. \\
\vspace{1.5em}
Группа \hrulefill ИУ7-104\\
\vspace{1.5em}
Преподаватель \hrulefill Ткачев С.Б.\\
\vspace{1.5em}
\end{flushleft}

%\vspace{\fill}
\vspace{13em}

\end{titlepage}


\newpage
\section{Описание задачи}
 
Функция: $y = \arcsin \left(\dfrac{35x^2 - 30x + 9}{20}\right) + \cos \left( \dfrac{10x^3 + 185x^2 + 340x + 103}{50x^2 + 100x + 30}\right)+0.5$ \\
Отрезок поиска: $x \in [0, 1]$

\subsection{Лабораторная работа \textnumero 1}
Метод поразрядного поиска.
\begin{table}[!ht]
  \begin{tabular}{|c|c|c|c|c|}
  \hline
  N & заданная точность & количество вычислений функции & $x^{\star}$ & $f(x^{\star})$ \\
  \hline
  1 & 1e-2 & 20 & 4.140625e-001 & -3.224396e-001 \\
  \hline
  2 & 1e-4 & 35 & 4.177246e-001 & -3.224625e-001 \\
  \hline
  3 & 1e-6 & 51 & 4.176331e-001 & -3.224625e-001 \\
  \hline
  \end{tabular}
\end{table}

\subsection{Лабораторная работа \textnumero 2}
Метод золотого сечения.
\begin{table}[!ht]
  \begin{tabular}{|c|c|c|c|c|}
  \hline
  N & заданная точность & количество вычислений функции & $x^{\star}$ & $f(x^{\star})$ \\
  \hline
  1 & 1e-2 & 11 & 4.179607e-001 & -3.224624e-001 \\
  \hline
  2 & 1e-4 & 20 & 4.176145e-001 & -3.224625e-001 \\
  \hline
  3 & 1e-6 & 30 & 4.176343e-001 & -3.224625e-001 \\
  \hline
  \end{tabular}
\end{table}

\subsection{Лабораторная работа \textnumero 3}
Метод квадратичной  интерполяции в сочетании с методом золотого сечения.
\begin{table}[!ht]
  \begin{tabular}{|c|c|c|c|c|}
  \hline
  N & заданная точность & количество вычислений функции & $x^{\star}$ & $f(x^{\star})$ \\
  \hline
  1 & 1e-2 & 14 & 4.108223e-001 & -3.223791e-001 \\
  \hline
  2 & 1e-4 & 18 & 4.176343e-001 & -3.224625e-001 \\
  \hline
  3 & 1e-6 & 18 & 4.176343e-001 & -3.224625e-001
 \\
  \hline
  \end{tabular}
\end{table}


\subsection{Лабораторная работа \textnumero 4}
Модифицированный метод Ньютона с конечно-разностной апроксимацией производных.
\begin{table}[!ht]
  \begin{tabular}{|c|c|c|c|c|}
  \hline
  N & заданная точность & количество вычислений функции & $x^{\star}$ & $f(x^{\star})$ \\
  \hline
  1 & 1e-2 & 12 & 4.126301e-001 & -3.224175e-001 \\
  \hline
  2 & 1e-4 & 12 & 4.126301e-001 & -3.224175e-001 \\
  \hline
  3 & 1e-6 & 15 & 4.126345e-001 & -3.224176e-001
 \\
  \hline
  \end{tabular}
\end{table}

\newpage

\subsection{Сводная таблица для решения задачи при точности 1e-6}
\begin{table}[!ht]
  \begin{tabular}{|c|c|c|c|c|}
  \hline
  Метод & количество вычислений функции & $x^{\star}$ & $f(x^{\star})$ \\
  \hline
  1 & 51 & 4.176331e-001 & -3.224625e-001 \\
  \hline
  2 & 30 & 4.176343e-001 & -3.224625e-001 \\
  \hline
  3 & 18 & 4.176343e-001 & -3.224625e-001 \\
  \hline
  4 & 15 & 4.126345e-001 & -3.224176e-001 \\
  \hline
  \end{tabular}
\end{table} 
 

\end{document}
