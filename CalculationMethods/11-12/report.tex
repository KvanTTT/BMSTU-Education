\documentclass[a4paper,12pt]{article}

\usepackage[utf8x]{inputenc}
\usepackage[T2A]{fontenc}
% \usepackage[russian, english]{babel}
\usepackage[english,russian]{babel}

% Опционно, требует  apt-get install scalable-cyrfonts.*
% и удаления одной строчки в cyrtimes.sty
% Сточку не удалять!
% \usepackage{cyrtimes}

% Картнки и tikz
\usepackage{graphicx}

% Некоторая русификация.
\usepackage{misccorr}
\usepackage{amsmath}
\usepackage{indentfirst}
\renewcommand{\labelitemi}{\normalfont\bfseries{--}}
% Увы, костыль
% \addto\captionsrussian{\def\figurename{{\cyr\CYRR\cyri\cyrs\cyru\cyrn\cyro\cyrk}}}
 


% Листинги

\usepackage{listingsutf8} 
\lstset{ 
inputencoding=utf8/koi8-r,
basicstyle=\small,
numbers=left,
numberstyle=\small,
stepnumber=1,
showspaces=false,
showtabs=false,
frame=single,
tabsize=4,
% captionpos=b,
breaklines=true,
% breakatwhitespace=false,
% escapeinside={\%*}{*)}
showstringspaces=false
}
\renewcommand{\lstlistingname}{Листинг}


% Увы, поля придётся уменьшить из-за листингов.
\topmargin -3cm
\oddsidemargin -0.5cm
\evensidemargin -0.5cm
\textwidth 18cm
\textheight 27cm

\sloppy

% Для исходного кода в тексте
\newcommand{\Code}[1]{\texttt{#1}}



\title{Отчёт по лабораторным работам \textnumero 11 --- 12.}
\author{Часть 4. Минимизация функций двух переменных при наличии ограничений. \\ Кочуркин~И.~А. ИУ7--104 \\ Вариант 8.}
\begin{document}

\maketitle

\section{Задание}
Для квадратичной функции $f$ задать три ограничения (два линейных и одно нелинейное) в виде нестрогих неравенств, чтобы:
\begin{enumerate}
\item допустимое множество было выпуклым;
\item точка минимума квадратичной функции не принадлежала допустимому множеству.
\end{enumerate}

$y = 2 x_1^2 + 5 x_2^2 - 4 x_1 x_2 - 4\sqrt{5} x_1 + 4 \sqrt{5} x_2 + 4$

\section{Решение}

\subsection{Выбранные ограничения}
\[
\left\{ 
\begin{array}{l}
  -x_2 - 1 \leq 0 \\
  3 x_1 + 3 x_2 - 1 \leq 0 \\
  x_1^2 + x_2^2 - 4 \leq 0 \\
\end{array} \right.
\]
Стартовая точка $x^0  = [x^0_1, x^0_2] = [1; 1]$. Точность равна $\pm0.001$.


% \tabularnewline 

\subsection{Минимизация квадратичной функции $f$}
\begin{table}[!ht]
  \begin{tabular}{|p{45mm}|c|c|c|c|}
  \hline
  \centering{Метод} & Кол-во вычисл. ф-ии & $x^{\star}$ & $f(x^{\star})$ \\
  \hline
  штрафных функций & 220 & (1.0252186; -0.6918790) & -4.0252498 \\
  \hline
  барьерных функций & 1080 & (1.0115953; -0.6782286) & -4.0232526 \\
  \hline
  \end{tabular}
\end{table}

\end{document}
